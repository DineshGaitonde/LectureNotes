\documentclass[12pt]{article}
% We can write notes using the percent symbol!
% The first line above is to announce we are beginning a document, an article in 
% this case, and we want the default font size to be 12pt
\usepackage[utf8]{inputenc}
% This is a package to accept utf8 input.  I normally do not use it in my 
% documents, but it was here by default in Overleaf.
\usepackage{pgfplots}
\usepackage{amsmath}
\usepackage{amssymb}
\usepackage{amsthm}
% These three packages are from the American Mathematical Society and includes 
% all of the important symbols and operations 
\usepackage{fullpage}
% By default, an article has some vary large margins to fit the smaller page 
% format.  This allows us to use more standard margins.

\setlength{\parskip}{1em}
% This gives us a full line break when we write a new paragraph

\begin{document}
% Once we have all of our packages and setting announced, we need to begin our 
% document.  You will notice that at the end of the writing there is an end 
% document statements.  Many options use this begin and end syntax.

\begin{center}
    \Large Statistical Mechanics \\
    Instructor - Prof. V. Balakrishnan
\end{center}

\section{Lecture 22 - Microcanonical Ensemble}

\subsection{Energy Momentum Relationship}
Consider a free particle with energy $E$. The non-relativistic energy momentum
relationship for such a particle is given by
$\begin{aligned}
    E & = \frac{p^2}{2m} \\
\end{aligned}$

One derives this relationship by considering the canonical relativistic 
relationship
\begin{equation} \label{eqn:emc2}
\begin{split}
E^2 & = m^2c^4 + p^2c^2 \\
& = m^2c^4\l( 1 + \frac{p^2}{m^2c^2} \r) \\
E & = mc^2\sqrt{\l( 1 + \frac{p^2}{m^2c^2}\r)}
\end{split}
\end{equation}

We consider particles where the term $\frac{p^2}{2m}$ is very small. If we 
expand RHS in equation~\ref{eqn:emc2} and only consider the linear term, we get 
the relationship

$\begin{aligned}
    E & = mc^2 + \frac{p^2}{2m}. \\
\end{aligned}$

Since the term $mc^2$ is simply the rest energy of the particle and does not 
change, we use the relationship $E = \frac{p^2}{2m}$

\subsection{Number of States For One Particle}
We are given that the energy of a single free particle is less than or equal to 
$\epsilon$. We want to compute $\phi (\epsilon)$ - the number of microstates the
particle could be in given that it is in some volume $V$ and within the energy 
constraint. The phase space for the particle are all (r, p) (position and 
momentum vectors) that are consistent with the constraints. That is, 

$\begin{aligned}
Phase\ Space\ Volume & = \int_{V}d^3r\int_{0}^{\sqrt{2m\epsilon}}d^3p \\
 \end{aligned}$

Since we cannot ask for momentum and position to accuracy less than order of 
$\hbar$, the number of states in this phase space volume - $\phi(\epsilon)$ is
given by

\begin{equation}\label{eqn:phi}
\begin{split}
\phi \l( \epsilon \r) & \propto \frac{V}{\hbar} \epsilon^\frac{3}{2}.
\end{split}
\end{equation}

It is instructive to understand the nature of the relationship in 
equation~\ref{eqn:phi}. The number of states a particle with energy no more than
$\epsilon$ can be in increases by $\frac{3}{2}$ power. The numerator $3$ comes 
from the fact that we are operating in $3$ dimensions. The denominator $2$ comes
from the fact that energy momentum relationship is quadratic.

The number of microstates in an energy shell of $d\epsilon$ can be expressed as
\begin{equation*}
\begin{split}
\phi \l( \epsilon + d\epsilon \r) -\phi\l(\epsilon\r) & = 
\frac{\partial \phi}{\partial \epsilon}d\epsilon
\end{split}
\end{equation*}

If we define $\frac{\partial \phi}{\partial \epsilon}$ as density of states
$\rho(\epsilon)$, then the number of states in a $d\epsilon$ shell is given by

$\begin{aligned}
\phi \l( \epsilon + d\epsilon \r) - \phi\l(\epsilon\r) & = 
\rho \l( \epsilon \r)d\epsilon \\
 \end{aligned}$

From equation~\ref{eqn:phi}, we can see that
\begin{equation}
\begin{split}
\rho \l( \epsilon \r) & \propto \frac{V}{\hbar}\epsilon^\frac{1}{2}
\end{split}
\end{equation}

Note that this dependence of $\rho$ on energy is true only in 3D. In 2D, $\rho$
is a constant. This is especially important for Quantum Hall effect where 
electrons are confined to a plane using an intense magnetic field. 

For bound quantum states, the energy spectrum is discrete. So one has to be
careful in defining $\rho$ as the partial derivative is not defined. But as
energies increase, the ladder spectrum begins to approximate the continuum. Hence
at large energies (large quantum numbers), $\rho$ makes sense.


\subsection{Number of States For Collection of Particles}

Consider a system of {\em free, non interacting} $N_{tot}$ particles enclosed in
some volume. The total energy of all the particles is $E_{tot}$. The number of 
microstates this collection could be in is represented as $\Omega\l( E_{tot}\r)$.
Consider now a partition of the original system into two partitions $A$ and $B$.
The energy of partition $A$ ($E$) and partition $B$ ($E^\prime$). $E + E^\prime$
equals $E_{tot}$ as we assume there is no interaction between the particles. 
Similarly, the number of particles in A (B) are $N$ ($N^\prime$) such that
$N + N^\prime$ equals $N_{tot}$.

\begin{equation*}
\begin{split}
E & = \sum_{i=1}{N}\epsilon_i \\
\end{split}
\end{equation*}

Assuming all the particles are independent, the total number of states of partition
$A$ is simply the product of the microstates of each particle
\begin{equation*}
\begin{split}
\Omega\l ( E \r) & = \prod_{i=1}^{N}\rho \l( \epsilon_i \r) \\
& \propto \prod_{i=1}^{N} \epsilon_i^\frac{1}{2}
\end{split}
\end{equation*} 

Assuming that the most likely situation is that each particle is close to the 
average energy, we get
\begin{equation*}
\begin{split}
\Omega\l ( E \r) & = \prod_{i=1}^{N}\left(\frac{E}{N}\right)^\frac{1}{2} \\
& \propto \left(\frac{E}{N}\right)^\frac{N}{2} \\
& \propto E^{N/2 -log_E{N}}
\end{split}
\end{equation*}

In terms of order of magnitudes (assuming $N$ is very large), we get the very 
important relationship ($\alpha$ is some constant) that the number of states
grows exponentionally with $N$
\begin{equation}
\begin{split}
\Omega\l ( E \r) & \propto E^{\alpha N} \\
\end{split}
\end{equation}


\end{document}
